\clearpage
\thispagestyle{empty}

{\centering
\textbf{
    UNIVERSIT\'E LIBRE DE BRUXELLES\\
    Faculté des Sciences\\
    Département d'Informatique
}

\vfill{}

{\Huge Churn Prediction and Causal Analysis \\ on Telecom Customer Data\\}
\vspace{0.5cm}
{\LARGE Théo Verhelst}

\vfill{}

\includegraphics[width=5cm]{figures/seal.pdf}

\vfill{}

\begin{flushleft}
    {\large
    \textbf{Promotor :}     \hfill{} Mémoire présenté en vue de\\
    Prof. Gianluca Bontempi \hfill{} l’obtention du grade de \\
                            \hfill{} Master en Sciences Informatiques
    }
\end{flushleft}

\enlargethispage{1.5cm}
\vfill{}

\textbf{Academic year 2018~-~2019}\par
}

\cleartorecto
\thispagestyle{plain}

\begin{vplace}
\begin{abstract}

Telecommunication companies are evolving in a highly competitive market where
attracting new customers is more expensive than retaining existing ones.
Retention campaigns can be used to prevent customer churn, but this requires
efficient churn prediction models. Such prediction is a difficult problem,
involving large amount of data, non-linearity, imbalance and overlap between
churners and non-churners. In this master thesis, we approach this problem with
Orange Belgium customer data. A descriptive analysis of the dataset is
conducted, and predictive modeling of churn is achieved with a random forest
classifier. The large class imbalance between the two classes is handled with
the Easy Ensemble algorithm. We assess the impact of different data
preprocessing techniques such as feature selection and creation of new features.
The directionality of the impact of variables on churn is estimated through a
sensitivity analysis. We observe that feature selection can be used to reduce
computation time and memory requirements, but the new variables we engineered do
not improve performance. Important variables include, non-exhaustively: the
tenure, the tariff plan, the number of calls, and the data usage. We explore the
application of data-driven causal inference, which allows to infer causal
relationships between variables purely from observational data. More
specifically, a causal Bayesian network, two methods of Markov blanket
inference, two causal filters and a supervised method are applied. Supported by
the prior knowledge of experts at Orange Belgium, we conclude that the bill
shock and the wrong tariff plan positioning are likely hypotheses of churn.

\end{abstract}
\end{vplace}

\clearpage
\thispagestyle{plain}
\begin{vplace}
\begin{otherlanguage}{french}
\begin{abstract}

Les compagnies de télécommunication évoluent dans un marché hautement compétitif
où attirer de nouveaux clients est plus coûteux que de retenir les clients déjà
présents. Des campagnes de rétention peuvent être utilisées pour réduire la
résiliation des clients, mais cela nécessite des modèles de prédiction
efficaces. La prédiction de résiliation est un problème difficile, impliquant de
grands quantités de données, des relations non-linéaires, des classes
non-équilibrées et avec une large superposition inter-classe. Dans ce mémoire de
master, nous abordons le problème de prédiction de résiliation avec des données
client de Orange Belgium. Une analyse descriptive du jeu de données est
effectuée, et une modélisation prédictive de la résiliation est obtenue en
utilisant un classificateur random forest. Le non-équilibrage entre les classes
est pris en compte avec l'algorithme Easy Ensemble. Nous évaluons l'impact de
différentes techniques de prétraitement de données, telles que la sélection de
variables et la création de nouvelles variables. La direction de l'impact des
variables sur la résiliation est déduite avec une analyse de sensibilité. On
constate que la sélection de variables permet de réduire les besoins en temps de
calcul et en mémoire, mais les nouvelles variables que nous avons conçues
n'améliorent pas les performances. Les variables importantes sont,
non-exhaustivement: la date d'inscription, le plan tarifaire, le nombre moyen
d'appels, et la consommation de données. Nous explorons l'utilisation de
l'inférence causale, qui permet de comprendre les liens de causalité entre
différentes variables à partir de données d'observation. Plus spécifiquement,
nous utilisons un réseau bayésien causal, une méthode d'inférence de couverture
de Markov, deux filtres causaux et une méthode supervisée. En prenant en compte
les connaissances préalables d'experts chez Orange Belgium, nous concluons
qu'une \og~mauvaise surprise~\fg{} à la réception de la facture et un mauvais
positionnement de plan tarifaire sont deux causes probables de résiliation.

\end{abstract}
\end{otherlanguage}
\end{vplace}


\clearpage
\chapter*{Acknowledgments}

First and foremost, I would like to thank Prof. Gianluca Bontempi for allowing
me to work with him on this subject and for his unfailing support and
dedication. It is only after these two years that I am able to realize how much
I learned on scientific research and communication with him.

In addition, I am grateful to Olivier Caelen, Jean-Christophe Dewitte, and the
rest of the data science team at Orange for introducing me to their work, for
providing all the resources necessary for this thesis, and for the friendly
atmosphere in the Orange office. In particular, the work of Pierre Brogniet on
churn prediction during his internship brought very valuable and complementary
knowledge to my work.

My special thanks go to my parents, my partner and the rest of my family for
their love and support during these two years and before.

I would like to thank my fellow students at the ULB for the mutual support and
the uncountable science-enthusiastic discussions, fostering my motivation to
pursue scientific research.

The attentive reading of Bertrand Lebichot and Arnaud Pollaris was greatly
appreciated, as they gave me insightful remarks to make my text clearer and more
comprehensible.

Lastly, I would like to express my gratitude to Prof. Tom Lenaerts and Prof.
Bernard Manderick for accepting to evaluate my thesis. Moreover, Prof. Tom
Lenaerts also allowed me to take part in an Erasmus program at Southampton
University, which played a significant role in my academic formation.

\newgeometry{right=2.5cm,top=0.5cm,bottom=2.5cm,left=3.5cm}
\tableofcontents*
\restoregeometry
