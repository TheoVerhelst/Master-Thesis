\clearpage
\thispagestyle{empty}

{\centering
\textbf{
    UNIVERSIT\'E LIBRE DE BRUXELLES\\
    Faculté des Sciences\\
    Département d'Informatique
}

\vfill{}

{\Huge Churn Prediction and Causal Analysis \\ on Telecom Customer Data\\}
\vspace{0.5cm}
{\LARGE Théo Verhelst}

\vfill{}

\includegraphics[width=5cm]{figures/seal.pdf}

\vfill{}

\begin{flushleft}
    {\large
    \textbf{Promotor :}    \hfill{} Mémoire présenté en vue de\\
    Prof. Gianluca Bontempi \hfill{} l’obtention du grade de \\
                            \hfill{} Master en Sciences Informatiques
    }
\end{flushleft}

\enlargethispage{1.5cm}
\vfill{}

\textbf{Academic year 2018~-~2019}\par
}

\cleartorecto

\thispagestyle{plain}
\begin{vplace}
\begin{abstract}

Telecommunication companies are evolving in a highly competitive market where
attracting new customers is more expensive than retaining existing ones.
Retention campaigns can be used to reduce the incentives of churn, but this
requires efficient churn prediction models. In this master thesis, we approach
this problem with Orange Belgium customer data. A descriptive analysis of the
dataset is conducted, and predictive modelling of churn is achieved with a random
forest classifier and the Easy Ensemble algorithm. We assess the impact of
different data preprocessing techniques such as feature selection and creation
of new features. The directionality of the impact of variables on churn is
estimated through a sensitivity analysis. Also, we explore the application of
data-driven causal inference, which allows to infer causal relationships between
variables purely from observational data.

\end{abstract}
\vfill{}
\begin{otherlanguage}{french}
\begin{abstract}

Les compagnies de télécommunication évoluent dans un marché hautement compétitif
où attirer de nouveaux clients est plus coûteux que retenir les clients déjà
présents. Des campagnes de rétention peuvent être utilisées pour réduire le taux
de résiliation, mais cela nécessite des modèles de prédiction de résiliation
efficaces. Dans ce mémoire de master, nous abordons le problème de prédiction de
résiliation avec des données client de Orange Belgium. Une analyse descriptive
du jeu de données est effectuée, et une modélisation prédictive de la
résiliation est obtenue en utilisant un classificateur random forest et
l'algorithme Easy Ensemble. Nous évaluons l'impact de différentes techniques de
prétraitement de données, telles que la sélection de variable et la création de
nouvelles variables. La directionalité de l'impact des variables sur la
résiliation est déduite avec une analyse de sensibilité. Nous explorons
également l'utilisation de l'inférence causale, qui permet de comprendre les
liens de causalité entre différentes variables à partir de données
d'observation.

\end{abstract}
\end{otherlanguage}
\end{vplace}


\clearpage
\chapter*{Acknowledgements}

\dots

\clearpage

\tableofcontents*
%\listoffigures
%\listoftables
